
\documentclass{article}
\usepackage{amssymb}
\usepackage{amsmath}
\usepackage[margin=0.5in]{geometry}

\begin{document}

\hspace*{2em}{\large (a)}\\
\hspace*{3em}We know that $c_1 < c_2$, with $G_1 = (V, E_1)$ and $G_2 = (V, E_2)$ where $E_i = \{ uv \in E : c(uv) = c_i \}$ and $p_1$ and $p_2$ connected components\\
\hspace*{3em} This means that $G_1$ will include the vertices which have been assigned the lower value $c_1$, and $G_2$ will include the vertices
which have been assigned the greater value. We are trying to prove $min-cost(G) = [c_1(n - p_1) + c_2(p_1 - 1)]$ \\
\hspace*{3em}By the MST cut property, for any cut of a graph, the minimum-weight edge crossing the cut must be included in every MST.\\
\hspace*{3em}In our case, $G_1$ has $p_1$ connected components and only edges of cost $c_1$. Inside each component, all vertices can be connected using only $c_1$ edges. There are $n - p_1$ edges of cost $c_1$ in the MST because each of the $p_1$ components of $G_1$ with $k_i$ vertices contributes $k_i - 1$ edges, i.e., $\sum_{i=1}^{p_1} (k_i - 1) = n - p_1$. \\
\hspace*{3em}To connect the $p_1$ components into a single spanning tree, we need exactly $p_1 - 1$ edges. These edges cannot be from $G_1$ (since components are disconnected in $G_1$), so they must be from $G_2$ with cost $c_2$.\\
\hspace*{3em}Therefore, the MST uses $n - p_1$ edges of cost $c_1$ and $p_1 - 1$ edges of cost $c_2$, giving total cost: $min\text{-}cost(G) = c_1 (n - p_1) + c_2 (p_1 - 1)$.

\hspace*{2em}{\large (b)}\\
\hspace*{3em}First, we create $G_1$ and $G_2$ as the problem states, with $G_1 = (V, E_1)$ and $G_2 = (V, E_2)$. Thus, we can separate the problem into finding trees with all edge values equal to $c_1$ and then connecting them with $c_2$ cost edges.\\
\hspace*{3em}To obtain all the $c_1$ cost edges needed for the MST, we can run a DFS or BFS algorithm on all connected components of $G_1$. Since all costs are the same, any resulting tree is optimal for our MST. While doing this step, it is important to mark each component with an ID, in order to set up the next step using Union-Find.\\
\hspace*{3em}We iterate through $G_2$'s edges. For each $uv \in G_2$, if $find(u) \neq find(v)$ then we perform $union(u,v)$ and add $uv$ to the MST. Exactly $p_1 - 1$ edges from $G_2$ are needed to connect all components.\\
\hspace*{3em}\begin{itemize}
\item Creating $G_1$ and $G_2$ happens in $O(n+m)$.
\item Iterating through $G_1$'s connected components happens in $O(n)$.
\item Iterating through $G_2$'s edges and performing the Union-Find operations happens in $O(m \cdot \alpha(n))$ amortized time, with $m$ as an upper bound for the size of $E_2$ and $\alpha(n)$ denoting the inverse Ackermann function.
\end{itemize}
\hspace*{3em}Given that the inverse Ackermann function is upper bounded by 5, we can say that the suggested algorithm runs in $O(n+m)$ in practice.
\end{document}
